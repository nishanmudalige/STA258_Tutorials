% Options for packages loaded elsewhere
% Options for packages loaded elsewhere
\PassOptionsToPackage{unicode}{hyperref}
\PassOptionsToPackage{hyphens}{url}
\PassOptionsToPackage{dvipsnames,svgnames,x11names}{xcolor}
%
\documentclass[
  letterpaper,
  DIV=11,
  numbers=noendperiod]{scrartcl}
\usepackage{xcolor}
\usepackage{amsmath,amssymb}
\setcounter{secnumdepth}{-\maxdimen} % remove section numbering
\usepackage{iftex}
\ifPDFTeX
  \usepackage[T1]{fontenc}
  \usepackage[utf8]{inputenc}
  \usepackage{textcomp} % provide euro and other symbols
\else % if luatex or xetex
  \usepackage{unicode-math} % this also loads fontspec
  \defaultfontfeatures{Scale=MatchLowercase}
  \defaultfontfeatures[\rmfamily]{Ligatures=TeX,Scale=1}
\fi
\usepackage{lmodern}
\ifPDFTeX\else
  % xetex/luatex font selection
\fi
% Use upquote if available, for straight quotes in verbatim environments
\IfFileExists{upquote.sty}{\usepackage{upquote}}{}
\IfFileExists{microtype.sty}{% use microtype if available
  \usepackage[]{microtype}
  \UseMicrotypeSet[protrusion]{basicmath} % disable protrusion for tt fonts
}{}
\makeatletter
\@ifundefined{KOMAClassName}{% if non-KOMA class
  \IfFileExists{parskip.sty}{%
    \usepackage{parskip}
  }{% else
    \setlength{\parindent}{0pt}
    \setlength{\parskip}{6pt plus 2pt minus 1pt}}
}{% if KOMA class
  \KOMAoptions{parskip=half}}
\makeatother
% Make \paragraph and \subparagraph free-standing
\makeatletter
\ifx\paragraph\undefined\else
  \let\oldparagraph\paragraph
  \renewcommand{\paragraph}{
    \@ifstar
      \xxxParagraphStar
      \xxxParagraphNoStar
  }
  \newcommand{\xxxParagraphStar}[1]{\oldparagraph*{#1}\mbox{}}
  \newcommand{\xxxParagraphNoStar}[1]{\oldparagraph{#1}\mbox{}}
\fi
\ifx\subparagraph\undefined\else
  \let\oldsubparagraph\subparagraph
  \renewcommand{\subparagraph}{
    \@ifstar
      \xxxSubParagraphStar
      \xxxSubParagraphNoStar
  }
  \newcommand{\xxxSubParagraphStar}[1]{\oldsubparagraph*{#1}\mbox{}}
  \newcommand{\xxxSubParagraphNoStar}[1]{\oldsubparagraph{#1}\mbox{}}
\fi
\makeatother

\usepackage{color}
\usepackage{fancyvrb}
\newcommand{\VerbBar}{|}
\newcommand{\VERB}{\Verb[commandchars=\\\{\}]}
\DefineVerbatimEnvironment{Highlighting}{Verbatim}{commandchars=\\\{\}}
% Add ',fontsize=\small' for more characters per line
\usepackage{framed}
\definecolor{shadecolor}{RGB}{241,243,245}
\newenvironment{Shaded}{\begin{snugshade}}{\end{snugshade}}
\newcommand{\AlertTok}[1]{\textcolor[rgb]{0.68,0.00,0.00}{#1}}
\newcommand{\AnnotationTok}[1]{\textcolor[rgb]{0.37,0.37,0.37}{#1}}
\newcommand{\AttributeTok}[1]{\textcolor[rgb]{0.40,0.45,0.13}{#1}}
\newcommand{\BaseNTok}[1]{\textcolor[rgb]{0.68,0.00,0.00}{#1}}
\newcommand{\BuiltInTok}[1]{\textcolor[rgb]{0.00,0.23,0.31}{#1}}
\newcommand{\CharTok}[1]{\textcolor[rgb]{0.13,0.47,0.30}{#1}}
\newcommand{\CommentTok}[1]{\textcolor[rgb]{0.37,0.37,0.37}{#1}}
\newcommand{\CommentVarTok}[1]{\textcolor[rgb]{0.37,0.37,0.37}{\textit{#1}}}
\newcommand{\ConstantTok}[1]{\textcolor[rgb]{0.56,0.35,0.01}{#1}}
\newcommand{\ControlFlowTok}[1]{\textcolor[rgb]{0.00,0.23,0.31}{\textbf{#1}}}
\newcommand{\DataTypeTok}[1]{\textcolor[rgb]{0.68,0.00,0.00}{#1}}
\newcommand{\DecValTok}[1]{\textcolor[rgb]{0.68,0.00,0.00}{#1}}
\newcommand{\DocumentationTok}[1]{\textcolor[rgb]{0.37,0.37,0.37}{\textit{#1}}}
\newcommand{\ErrorTok}[1]{\textcolor[rgb]{0.68,0.00,0.00}{#1}}
\newcommand{\ExtensionTok}[1]{\textcolor[rgb]{0.00,0.23,0.31}{#1}}
\newcommand{\FloatTok}[1]{\textcolor[rgb]{0.68,0.00,0.00}{#1}}
\newcommand{\FunctionTok}[1]{\textcolor[rgb]{0.28,0.35,0.67}{#1}}
\newcommand{\ImportTok}[1]{\textcolor[rgb]{0.00,0.46,0.62}{#1}}
\newcommand{\InformationTok}[1]{\textcolor[rgb]{0.37,0.37,0.37}{#1}}
\newcommand{\KeywordTok}[1]{\textcolor[rgb]{0.00,0.23,0.31}{\textbf{#1}}}
\newcommand{\NormalTok}[1]{\textcolor[rgb]{0.00,0.23,0.31}{#1}}
\newcommand{\OperatorTok}[1]{\textcolor[rgb]{0.37,0.37,0.37}{#1}}
\newcommand{\OtherTok}[1]{\textcolor[rgb]{0.00,0.23,0.31}{#1}}
\newcommand{\PreprocessorTok}[1]{\textcolor[rgb]{0.68,0.00,0.00}{#1}}
\newcommand{\RegionMarkerTok}[1]{\textcolor[rgb]{0.00,0.23,0.31}{#1}}
\newcommand{\SpecialCharTok}[1]{\textcolor[rgb]{0.37,0.37,0.37}{#1}}
\newcommand{\SpecialStringTok}[1]{\textcolor[rgb]{0.13,0.47,0.30}{#1}}
\newcommand{\StringTok}[1]{\textcolor[rgb]{0.13,0.47,0.30}{#1}}
\newcommand{\VariableTok}[1]{\textcolor[rgb]{0.07,0.07,0.07}{#1}}
\newcommand{\VerbatimStringTok}[1]{\textcolor[rgb]{0.13,0.47,0.30}{#1}}
\newcommand{\WarningTok}[1]{\textcolor[rgb]{0.37,0.37,0.37}{\textit{#1}}}

\usepackage{longtable,booktabs,array}
\usepackage{calc} % for calculating minipage widths
% Correct order of tables after \paragraph or \subparagraph
\usepackage{etoolbox}
\makeatletter
\patchcmd\longtable{\par}{\if@noskipsec\mbox{}\fi\par}{}{}
\makeatother
% Allow footnotes in longtable head/foot
\IfFileExists{footnotehyper.sty}{\usepackage{footnotehyper}}{\usepackage{footnote}}
\makesavenoteenv{longtable}
\usepackage{graphicx}
\makeatletter
\newsavebox\pandoc@box
\newcommand*\pandocbounded[1]{% scales image to fit in text height/width
  \sbox\pandoc@box{#1}%
  \Gscale@div\@tempa{\textheight}{\dimexpr\ht\pandoc@box+\dp\pandoc@box\relax}%
  \Gscale@div\@tempb{\linewidth}{\wd\pandoc@box}%
  \ifdim\@tempb\p@<\@tempa\p@\let\@tempa\@tempb\fi% select the smaller of both
  \ifdim\@tempa\p@<\p@\scalebox{\@tempa}{\usebox\pandoc@box}%
  \else\usebox{\pandoc@box}%
  \fi%
}
% Set default figure placement to htbp
\def\fps@figure{htbp}
\makeatother





\setlength{\emergencystretch}{3em} % prevent overfull lines

\providecommand{\tightlist}{%
  \setlength{\itemsep}{0pt}\setlength{\parskip}{0pt}}



 


\KOMAoption{captions}{tableheading}
\makeatletter
\@ifpackageloaded{tcolorbox}{}{\usepackage[skins,breakable]{tcolorbox}}
\@ifpackageloaded{fontawesome5}{}{\usepackage{fontawesome5}}
\definecolor{quarto-callout-color}{HTML}{909090}
\definecolor{quarto-callout-note-color}{HTML}{0758E5}
\definecolor{quarto-callout-important-color}{HTML}{CC1914}
\definecolor{quarto-callout-warning-color}{HTML}{EB9113}
\definecolor{quarto-callout-tip-color}{HTML}{00A047}
\definecolor{quarto-callout-caution-color}{HTML}{FC5300}
\definecolor{quarto-callout-color-frame}{HTML}{acacac}
\definecolor{quarto-callout-note-color-frame}{HTML}{4582ec}
\definecolor{quarto-callout-important-color-frame}{HTML}{d9534f}
\definecolor{quarto-callout-warning-color-frame}{HTML}{f0ad4e}
\definecolor{quarto-callout-tip-color-frame}{HTML}{02b875}
\definecolor{quarto-callout-caution-color-frame}{HTML}{fd7e14}
\makeatother
\makeatletter
\@ifpackageloaded{caption}{}{\usepackage{caption}}
\AtBeginDocument{%
\ifdefined\contentsname
  \renewcommand*\contentsname{Table of contents}
\else
  \newcommand\contentsname{Table of contents}
\fi
\ifdefined\listfigurename
  \renewcommand*\listfigurename{List of Figures}
\else
  \newcommand\listfigurename{List of Figures}
\fi
\ifdefined\listtablename
  \renewcommand*\listtablename{List of Tables}
\else
  \newcommand\listtablename{List of Tables}
\fi
\ifdefined\figurename
  \renewcommand*\figurename{Figure}
\else
  \newcommand\figurename{Figure}
\fi
\ifdefined\tablename
  \renewcommand*\tablename{Table}
\else
  \newcommand\tablename{Table}
\fi
}
\@ifpackageloaded{float}{}{\usepackage{float}}
\floatstyle{ruled}
\@ifundefined{c@chapter}{\newfloat{codelisting}{h}{lop}}{\newfloat{codelisting}{h}{lop}[chapter]}
\floatname{codelisting}{Listing}
\newcommand*\listoflistings{\listof{codelisting}{List of Listings}}
\usepackage{amsthm}
\theoremstyle{remark}
\AtBeginDocument{\renewcommand*{\proofname}{Proof}}
\newtheorem*{remark}{Remark}
\newtheorem*{solution}{Solution}
\newtheorem{refremark}{Remark}[section]
\newtheorem{refsolution}{Solution}[section]
\makeatother
\makeatletter
\makeatother
\makeatletter
\@ifpackageloaded{caption}{}{\usepackage{caption}}
\@ifpackageloaded{subcaption}{}{\usepackage{subcaption}}
\makeatother
\usepackage{bookmark}
\IfFileExists{xurl.sty}{\usepackage{xurl}}{} % add URL line breaks if available
\urlstyle{same}
\hypersetup{
  pdftitle={Tutorial 01: Introduction to R},
  colorlinks=true,
  linkcolor={blue},
  filecolor={Maroon},
  citecolor={Blue},
  urlcolor={Blue},
  pdfcreator={LaTeX via pandoc}}


\title{Tutorial 01: Introduction to R}
\author{}
\date{}
\begin{document}
\maketitle


\begin{Shaded}
\begin{Highlighting}[]
\NormalTok{\#| edit: false}
\NormalTok{\#| output: false}
\NormalTok{webr::install("gradethis", quiet = TRUE)}
\NormalTok{library(gradethis)}
\NormalTok{options(webr.exercise.checker = function(}
\NormalTok{  label, user\_code, solution\_code, check\_code, envir\_result, evaluate\_result,}
\NormalTok{  envir\_prep, last\_value, engine, stage, ...}
\NormalTok{) \{}
\NormalTok{  if (is.null(check\_code)) \{}
\NormalTok{    \# No grading code, so just skip grading}
\NormalTok{    invisible(NULL)}
\NormalTok{  \} else if (is.null(label)) \{}
\NormalTok{    list(}
\NormalTok{      correct = FALSE,}
\NormalTok{      type = "warning",}
\NormalTok{      message = "All exercises must have a label."}
\NormalTok{    )}
\NormalTok{  \} else if (is.null(solution\_code)) \{}
\NormalTok{    list(}
\NormalTok{      correct = FALSE,}
\NormalTok{      type = "warning",}
\NormalTok{      message = htmltools::tags$div(}
\NormalTok{        htmltools::tags$p("A problem occurred grading this exercise."),}
\NormalTok{        htmltools::tags$p(}
\NormalTok{          "No solution code was found. Note that grading exercises using the ",}
\NormalTok{          htmltools::tags$code("gradethis"),}
\NormalTok{          "package requires a model solution to be included in the document."}
\NormalTok{        )}
\NormalTok{      )}
\NormalTok{    )}
\NormalTok{  \} else \{}
\NormalTok{    gradethis::gradethis\_exercise\_checker(}
\NormalTok{      label = label, solution\_code = solution\_code, user\_code = user\_code,}
\NormalTok{      check\_code = check\_code, envir\_result = envir\_result,}
\NormalTok{      evaluate\_result = evaluate\_result, envir\_prep = envir\_prep,}
\NormalTok{      last\_value = last\_value, stage = stage, engine = engine)}
\NormalTok{  \}}
\NormalTok{\})}
\end{Highlighting}
\end{Shaded}

\subsection{Q1 --- Make the sum equal 10}\label{q1}

Replace \texttt{_} with a number so the expression evaluates to
\textbf{10}.

\begin{Shaded}
\begin{Highlighting}[]
\NormalTok{\#| setup: true}
\NormalTok{\#| exercise: ex\_sum}
\NormalTok{_ \textless{}{-} NA\_real\_}
\end{Highlighting}
\end{Shaded}

\begin{Shaded}
\begin{Highlighting}[]
\NormalTok{\#| exercise: ex\_sum}
\NormalTok{1 + 2 + _ + _}
\end{Highlighting}
\end{Shaded}

Replace the two _ with 2 numbers that add up to 7.

\begin{solution}
\leavevmode

\begin{Shaded}
\begin{Highlighting}[]
\NormalTok{\#| exercise: ex\_sum}
\NormalTok{\#| solution: true }
\NormalTok{1 + 2 + 3 + 4}
\end{Highlighting}
\end{Shaded}

\end{solution}

\begin{Shaded}
\begin{Highlighting}[]
\NormalTok{\#| exercise: ex\_sum}
\NormalTok{\#| check: true}
\NormalTok{gradethis::grade\_this(\{}
\NormalTok{  if (grepl("_", .user\_code, fixed = TRUE)) \{}
\NormalTok{    fail("Replace _ with a number.")}
\NormalTok{  \}}
\NormalTok{  \# Pass if the evaluated result is (numerically) 10}
\NormalTok{  if (is.numeric(.result) \&\& length(.result) == 1 \&\&}
\NormalTok{      isTRUE(all.equal(.result, 10))) \{}
\NormalTok{    pass("✅ Correct!")}
\NormalTok{  \} else \{}
\NormalTok{    fail("Not quite — try to make the expression evaluate to 10.")}
\NormalTok{  \}}
\NormalTok{\})}
\end{Highlighting}
\end{Shaded}

\subsection{Q2 --- Create a vector}\label{q2-create-a-vector}

Make a vector called \texttt{my\_vec} that contains the numbers
\textbf{5, 10, 15, 20}.

\begin{Shaded}
\begin{Highlighting}[]
\NormalTok{\#| exercise: ex\_vec}
\NormalTok{\#| exercise.lines: 2}
\NormalTok{\#| echo: false}
\NormalTok{my\_vec \textless{}{-} NULL}
\end{Highlighting}
\end{Shaded}

Use the c() function to combine numbers: c(1, 2, 3)

\begin{solution}
\leavevmode

\begin{Shaded}
\begin{Highlighting}[]
\NormalTok{\#| exercise: ex\_vec}
\NormalTok{\#| solution: true}
\NormalTok{my\_vec \textless{}{-} c(5, 10, 15, 20)}
\end{Highlighting}
\end{Shaded}

\end{solution}

\begin{Shaded}
\begin{Highlighting}[]
\NormalTok{\#| exercise: ex\_vec}
\NormalTok{\#| check: true}
\NormalTok{gradethis::grade\_this(\{}
\NormalTok{  if (!exists("my\_vec", envir = .envir\_result)) \{}
\NormalTok{    fail("You need to define \textasciigrave{}my\_vec\textasciigrave{}.")}
\NormalTok{  \}}
\NormalTok{  v \textless{}{-} get("my\_vec", envir = .envir\_result)}
\NormalTok{  if (!is.numeric(v)) \{}
\NormalTok{    fail("\textasciigrave{}my\_vec\textasciigrave{} should be numeric.")}
\NormalTok{  \}}
\NormalTok{  if (!identical(v, c(5, 10, 15, 20))) \{}
\NormalTok{    fail("\textasciigrave{}my\_vec\textasciigrave{} should contain exactly 5, 10, 15, 20 in that order.")}
\NormalTok{  \}}
\NormalTok{  pass("✅ Nice! Your vector is correct.")}
\NormalTok{\})}
\end{Highlighting}
\end{Shaded}

\subsection{Q3 --- Find the average}\label{q3-find-the-average}

Compute the mean of the vector \texttt{c(2,\ 4,\ 6,\ 8,\ 10)}.\\
Your answer should be stored in a variable called \texttt{avg\_val}.

\begin{Shaded}
\begin{Highlighting}[]
\NormalTok{\#| exercise: ex\_mean}
\NormalTok{\#| exercise.lines: 2}
\NormalTok{\#| echo: false}
\NormalTok{avg\_val \textless{}{-} NULL}
\end{Highlighting}
\end{Shaded}

Use the mean() function: mean(c(\ldots))

\begin{solution}
\leavevmode

\begin{Shaded}
\begin{Highlighting}[]
\NormalTok{\#| exercise: ex\_mean}
\NormalTok{\#| solution: true}
\NormalTok{avg\_val \textless{}{-} mean(c(2, 4, 6, 8, 10))}
\end{Highlighting}
\end{Shaded}

\end{solution}

\begin{Shaded}
\begin{Highlighting}[]
\NormalTok{\#| exercise: ex\_mean}
\NormalTok{\#| check: true}
\NormalTok{gradethis::grade\_this(\{}
\NormalTok{  if (!exists("avg\_val", envir = .envir\_result)) \{}
\NormalTok{    fail("You need to define \textasciigrave{}avg\_val\textasciigrave{}.")}
\NormalTok{  \}}
\NormalTok{  v \textless{}{-} get("avg\_val", envir = .envir\_result)}
\NormalTok{  if (!is.numeric(v) || length(v) != 1) \{}
\NormalTok{    fail("\textasciigrave{}avg\_val\textasciigrave{} should be a single numeric value.")}
\NormalTok{  \}}
\NormalTok{  if (!isTRUE(all.equal(v, 6, tol = 1e{-}8))) \{}
\NormalTok{    fail("Not quite — the mean of c(2,4,6,8,10) is 6.")}
\NormalTok{  \}}
\NormalTok{  pass("✅ Correct! You computed the mean successfully.")}
\NormalTok{\})}
\end{Highlighting}
\end{Shaded}

\subsection{Q4 --- Draw a histogram from ToothGrowth (type the full
call)}\label{q4-draw-a-histogram-from-toothgrowth-type-the-full-call}

Use exactly one of these numeric columns and nothing else: len or dose

Photo by The Humble Co. on Unsplash

\begin{tcolorbox}[enhanced jigsaw, bottomtitle=1mm, arc=.35mm, toprule=.15mm, colframe=quarto-callout-note-color-frame, breakable, opacitybacktitle=0.6, leftrule=.75mm, titlerule=0mm, coltitle=black, opacityback=0, toptitle=1mm, bottomrule=.15mm, colbacktitle=quarto-callout-note-color!10!white, rightrule=.15mm, title=\textcolor{quarto-callout-note-color}{\faInfo}\hspace{0.5em}{Preview}, left=2mm, colback=white]

Find a preview of the ToothGrowth dataset here:

\begin{Shaded}
\begin{Highlighting}[]
\NormalTok{\#| echo: true}
\NormalTok{head(ToothGrowth, 8)}
\end{Highlighting}
\end{Shaded}

\end{tcolorbox}

\begin{Shaded}
\begin{Highlighting}[]
\NormalTok{\#| exercise: ex\_hist\_exact}
\NormalTok{\# Type your answer on the next line (one command only):}
\end{Highlighting}
\end{Shaded}

Use the hist command.

\begin{solution}
\leavevmode

\begin{Shaded}
\begin{Highlighting}[]
\NormalTok{\#| exercise: ex\_hist\_exact}
\NormalTok{\#| solution: true}
\NormalTok{hist(ToothGrowth$len)}
\end{Highlighting}
\end{Shaded}

\end{solution}

\begin{Shaded}
\begin{Highlighting}[]
\NormalTok{\#| exercise: ex\_hist\_exact}
\NormalTok{\#| check: true}
\NormalTok{gradethis::grade\_this(\{}
\NormalTok{code \textless{}{-} .user\_code}
\NormalTok{lines \textless{}{-} strsplit(paste(code, collapse = "\textbackslash{}n"), "\textbackslash{}n", fixed = TRUE)[[1]]}
\NormalTok{lines \textless{}{-} sub("\#.*$", "", lines)}
\NormalTok{trim \textless{}{-} function(s) sub("\^{}[[:space:]]+|[[:space:]]+$", "", s)}
\NormalTok{lines \textless{}{-} trim(lines); lines \textless{}{-} lines[nzchar(lines)]}

\NormalTok{if (length(lines) == 0) fail("Type your answer; it can\textquotesingle{}t be empty.")}
\NormalTok{if (length(lines) \textgreater{} 1)  fail("Enter exactly one command.")}

\NormalTok{one \textless{}{-} lines[1]}
\NormalTok{if (!inherits(.result, "histogram")) fail("Your code should draw a histogram.")}

\NormalTok{expr \textless{}{-} try(parse(text = one)[[1]], silent = TRUE)}
\NormalTok{if (inherits(expr, "try{-}error")) fail("Your code must be a single valid R expression.")}
\NormalTok{if (!(is.call(expr) \&\& identical(as.character(expr[[1]]), "hist"))) fail("Use hist(...).")}

\NormalTok{args \textless{}{-} as.list(expr)[{-}1]}
\NormalTok{if (length(args) != 1L) fail("Pass exactly one argument to hist().")}

\NormalTok{arg\_txt \textless{}{-} paste0(deparse(args[[1]]), collapse = "")}
\NormalTok{allowed \textless{}{-} c("ToothGrowth$len","ToothGrowth$dose")}

\NormalTok{if (arg\_txt \%in\% allowed) pass("OK")}
\NormalTok{else if (arg\_txt == "ToothGrowth") fail("hist(ToothGrowth) won’t work—choose len or dose.")}
\NormalTok{else fail("Type exactly hist(ToothGrowth$len) or hist(ToothGrowth$dose).")}
\NormalTok{\})}
\end{Highlighting}
\end{Shaded}

\subsection{Q5 --- Draw a boxplot from ToothGrowth (type the full
call)}\label{q5-draw-a-boxplot-from-toothgrowth-type-the-full-call}

Use exactly one of these numeric columns and nothing else: len or dose

\begin{tcolorbox}[enhanced jigsaw, bottomtitle=1mm, arc=.35mm, toprule=.15mm, colframe=quarto-callout-note-color-frame, breakable, opacitybacktitle=0.6, leftrule=.75mm, titlerule=0mm, coltitle=black, opacityback=0, toptitle=1mm, bottomrule=.15mm, colbacktitle=quarto-callout-note-color!10!white, rightrule=.15mm, title=\textcolor{quarto-callout-note-color}{\faInfo}\hspace{0.5em}{Preview}, left=2mm, colback=white]

Find a preview of the ToothGrowth dataset here:

\begin{Shaded}
\begin{Highlighting}[]
\NormalTok{\#| echo: true}
\NormalTok{head(ToothGrowth, 8)}
\end{Highlighting}
\end{Shaded}

\end{tcolorbox}

\begin{Shaded}
\begin{Highlighting}[]
\NormalTok{\#| exercise: ex\_boxplot\_exact}
\NormalTok{\# Type your answer on the next line (one command only):}
\end{Highlighting}
\end{Shaded}

Use the boxplot command.

\begin{solution}
\leavevmode

\begin{Shaded}
\begin{Highlighting}[]
\NormalTok{\#| exercise: ex\_boxplot\_exact}
\NormalTok{\#| solution: true}
\NormalTok{boxplot(ToothGrowth$len)}
\end{Highlighting}
\end{Shaded}

\end{solution}

\begin{Shaded}
\begin{Highlighting}[]
\NormalTok{\#| exercise: ex\_boxplot\_exact}
\NormalTok{\#| check: true}
\NormalTok{gradethis::grade\_this(\{}
\NormalTok{code \textless{}{-} .user\_code}
\NormalTok{lines \textless{}{-} strsplit(paste(code, collapse = "\textbackslash{}n"), "\textbackslash{}n", fixed = TRUE)[[1]]}
\NormalTok{lines \textless{}{-} sub("\#.*$", "", lines)}
\NormalTok{trim \textless{}{-} function(s) sub("\^{}[[:space:]]+|[[:space:]]+$", "", s)}
\NormalTok{lines \textless{}{-} trim(lines); lines \textless{}{-} lines[nzchar(lines)]}

\NormalTok{if (length(lines) == 0) fail("Type your answer; it can\textquotesingle{}t be empty.")}
\NormalTok{if (length(lines) \textgreater{} 1)  fail("Enter exactly one command.")}

\NormalTok{one \textless{}{-} lines[1]}

\NormalTok{expr \textless{}{-} try(parse(text = one)[[1]], silent = TRUE)}
\NormalTok{if (inherits(expr, "try{-}error")) fail("Your code must be a single valid R expression.")}
\NormalTok{if (!(is.call(expr) \&\& identical(as.character(expr[[1]]), "boxplot"))) fail("Use boxplot(...).")}

\NormalTok{args \textless{}{-} as.list(expr)[{-}1]}
\NormalTok{if (length(args) != 1L) fail("Pass exactly one argument to boxplot().")}

\NormalTok{arg\_txt \textless{}{-} paste0(deparse(args[[1]]), collapse = "")}
\NormalTok{allowed \textless{}{-} c("ToothGrowth$len","ToothGrowth$dose")}

\NormalTok{if (arg\_txt \%in\% allowed) \{}
\NormalTok{if (!(is.list(.result) \&\& !is.null(.result$stats))) pass("OK (note: base boxplot())") else pass("OK")}
\NormalTok{\} else if (arg\_txt == "ToothGrowth") \{}
\NormalTok{fail("boxplot(ToothGrowth) won’t work—choose len or dose.")}
\NormalTok{\} else \{}
\NormalTok{fail("Type exactly boxplot(ToothGrowth$len) or boxplot(ToothGrowth$dose).")}
\NormalTok{\}}
\NormalTok{\})}
\end{Highlighting}
\end{Shaded}

\subsection{Q6 --- Add a color to the
histogram}\label{q6-add-a-color-to-the-histogram}

Re-draw a histogram of ToothGrowth\$len and set any bar color using the
col= argument.

\begin{Shaded}
\begin{Highlighting}[]
\NormalTok{\#| exercise: ex\_hist\_col}
\NormalTok{\# (one command)}
\end{Highlighting}
\end{Shaded}

add col as an additional argument here and give it a color of your
choice.

\begin{solution}
\leavevmode

\begin{Shaded}
\begin{Highlighting}[]
\NormalTok{\#| exercise: ex\_hist\_col}
\NormalTok{\#| solution: true}
\NormalTok{hist(ToothGrowth$len, col = "steelblue")}
\end{Highlighting}
\end{Shaded}

\end{solution}

\begin{Shaded}
\begin{Highlighting}[]
\NormalTok{\#| exercise: ex\_hist\_col}
\NormalTok{\#| check: true}
\NormalTok{gradethis::grade\_this(\{}
\NormalTok{line \textless{}{-} paste(.user\_code, collapse = "\textbackslash{}n")}
\NormalTok{if (!inherits(.result, "histogram")) fail("Use hist(...) on ToothGrowth$len.")}

\NormalTok{expr \textless{}{-} try(parse(text = line)[[1]], silent = TRUE)}
\NormalTok{if (inherits(expr, "try{-}error")) fail("Your code must be a single valid R expression.")}
\NormalTok{if (!(is.call(expr) \&\& identical(as.character(expr[[1]]), "hist"))) fail("Call hist(...).")}

\NormalTok{args \textless{}{-} as.list(expr)[{-}1]}
\NormalTok{arg1\_txt \textless{}{-} paste0(deparse(args[[1]]), collapse = "")}
\NormalTok{if (arg1\_txt != "ToothGrowth$len") fail("Use ToothGrowth$len as the data argument.")}

\NormalTok{has\_col \textless{}{-} any(names(args) == "col")}
\NormalTok{if (!has\_col) fail("Add a color with col = ...")}

\NormalTok{pass("OK")}
\NormalTok{\})}
\end{Highlighting}
\end{Shaded}

\subsection{Q7 --- Boxplot (make it
horizontal)}\label{q7-boxplot-make-it-horizontal}

Draw a boxplot of ToothGrowth\$dose and make it horizontal using
horizontal = TRUE as an argument.

\begin{Shaded}
\begin{Highlighting}[]
\NormalTok{\#| exercise: ex\_box\_h}
\NormalTok{\# (one command)}
\end{Highlighting}
\end{Shaded}

add horizontal as an argument just like col from the previous question

\begin{solution}
\leavevmode

\begin{Shaded}
\begin{Highlighting}[]
\NormalTok{\#| exercise: ex\_box\_h}
\NormalTok{\#| solution: true}
\NormalTok{boxplot(ToothGrowth$dose, horizontal = TRUE)}
\end{Highlighting}
\end{Shaded}

\end{solution}

\begin{Shaded}
\begin{Highlighting}[]
\NormalTok{\#| exercise: ex\_box\_h}
\NormalTok{\#| check: true}
\NormalTok{gradethis::grade\_this(\{}
\NormalTok{line \textless{}{-} paste(.user\_code, collapse = "\textbackslash{}n")}

\NormalTok{expr \textless{}{-} try(parse(text = line)[[1]], silent = TRUE)}
\NormalTok{if (inherits(expr, "try{-}error")) fail("Your code must be a single valid R expression.")}
\NormalTok{if (!(is.call(expr) \&\& identical(as.character(expr[[1]]), "boxplot"))) fail("Use boxplot(...).")}

\NormalTok{args \textless{}{-} as.list(expr)[{-}1]}
\NormalTok{if (length(args) \textless{} 1L) fail("Pass ToothGrowth$dose to boxplot().")}

\NormalTok{arg1\_txt \textless{}{-} paste0(deparse(args[[1]]), collapse = "")}
\NormalTok{if (arg1\_txt != "ToothGrowth$dose") fail("Use ToothGrowth$dose as the data argument.")}

\NormalTok{arg\_names \textless{}{-} names(args)}
\NormalTok{if (is.null(arg\_names)) arg\_names \textless{}{-} rep("", length(args))}
\NormalTok{idx \textless{}{-} which(arg\_names == "horizontal")}
\NormalTok{horiz\_ok \textless{}{-} FALSE}
\NormalTok{if (length(idx) \textgreater{} 0) \{}
\NormalTok{hval \textless{}{-} args[[idx[1]]]}
\NormalTok{horiz\_ok \textless{}{-} isTRUE(eval(hval))}
\NormalTok{\}}
\NormalTok{if (!horiz\_ok) fail("Add horizontal = TRUE.")}

\NormalTok{if (!(is.list(.result) \&\& !is.null(.result$stats))) \{}
\NormalTok{pass("OK (note: ensure you\textquotesingle{}re using base boxplot())")}
\NormalTok{\} else \{}
\NormalTok{pass("OK")}
\NormalTok{\}}
\NormalTok{\})}
\end{Highlighting}
\end{Shaded}

\subsection{Q8 --- Load a CSV as a
DataFrame}\label{q8-load-a-csv-as-a-dataframe}

Load iris.csv into a dataframe named df.

Photo by Andrew Small on Unsplash

\begin{Shaded}
\begin{Highlighting}[]
\NormalTok{\# Q9 SETUP — create iris.csv inside the webR filesystem}
\NormalTok{\#| setup: true}
\NormalTok{\#| exercise: ex\_csv\_load}
\NormalTok{write.csv(iris, "iris.csv", row.names = FALSE)}
\end{Highlighting}
\end{Shaded}

\begin{Shaded}
\begin{Highlighting}[]
\NormalTok{\#\# Q9 — Load a CSV as a data frame}
\NormalTok{\# Read data/iris.csv into a data frame named df}
\NormalTok{\# Read iris.csv into a data frame named df}
\NormalTok{\#| exercise: ex\_csv\_load}
\NormalTok{\#| exercise.lines: 3}
\NormalTok{\#| echo: false}
\NormalTok{df \textless{}{-} }
\end{Highlighting}
\end{Shaded}

Use read.csv here.

\begin{solution}
\leavevmode

\begin{Shaded}
\begin{Highlighting}[]
\NormalTok{\#| exercise: ex\_csv\_load}
\NormalTok{\#| solution: true}
\NormalTok{\#| files: ["data/iris.csv"]   \# solution runs in its own sandbox — mount again}
\NormalTok{df \textless{}{-} read.csv("iris.csv")}
\end{Highlighting}
\end{Shaded}

\end{solution}

\begin{Shaded}
\begin{Highlighting}[]
\NormalTok{\#| exercise: ex\_csv\_load}
\NormalTok{\#| check: true}
\NormalTok{gradethis::grade\_this(\{}
\NormalTok{  if (!exists("df", envir = .envir\_result)) fail("Create \textasciigrave{}df\textasciigrave{} with \textasciigrave{}read.csv(\textbackslash{}"iris.csv\textbackslash{}")\textasciigrave{}.")}
\NormalTok{  x \textless{}{-} get("df", envir = .envir\_result)}
\NormalTok{  if (!is.data.frame(x)) fail("\textasciigrave{}df\textasciigrave{} should be a data frame.")}
\NormalTok{  need \textless{}{-} c("Sepal.Length","Sepal.Width","Petal.Length","Petal.Width","Species")}
\NormalTok{  if (!all(need \%in\% names(x))) fail("\textasciigrave{}df\textasciigrave{} doesn\textquotesingle{}t look like the expected columns.")}
\NormalTok{  pass("✅ Loaded correctly!")}
\NormalTok{\})}
\end{Highlighting}
\end{Shaded}

\subsection{Q9 --- Count rows in
iris.csv}\label{q9-count-rows-in-iris.csv}

Count the number of rows in iris.csv and store them in the variable rows

\begin{Shaded}
\begin{Highlighting}[]
\NormalTok{\# Q10 setup}
\NormalTok{\#| setup: true}
\NormalTok{\#| exercise: ex\_csv\_nrows}
\NormalTok{write.csv(iris, "iris.csv", row.names = FALSE)}
\end{Highlighting}
\end{Shaded}

\begin{Shaded}
\begin{Highlighting}[]
\NormalTok{\#| exercise: ex\_csv\_nrows}
\NormalTok{\#| exercise.lines: 3}
\NormalTok{\#| echo: false}
\NormalTok{df \textless{}{-} read.csv("iris.csv")}
\NormalTok{rows \textless{}{-} }
\end{Highlighting}
\end{Shaded}

Use nrow.

\begin{solution}
\leavevmode

\begin{Shaded}
\begin{Highlighting}[]
\NormalTok{\#| exercise: ex\_csv\_nrows}
\NormalTok{\#| solution: true}
\NormalTok{\#| files: ["data/iris.csv"]   \# solution runs in its own sandbox — mount again}
\NormalTok{df \textless{}{-} read.csv("iris.csv")}
\NormalTok{rows \textless{}{-} nrow(df)}
\end{Highlighting}
\end{Shaded}

\end{solution}

\begin{Shaded}
\begin{Highlighting}[]
\NormalTok{\#| exercise: ex\_csv\_nrows}
\NormalTok{\#| check: true}
\NormalTok{gradethis::grade\_this(\{}
\NormalTok{  stopifnot(exists("rows", envir=.envir\_result))}
\NormalTok{  x \textless{}{-} get("rows", envir=.envir\_result)}
\NormalTok{  if (!is.numeric(x) || length(x)!=1) fail("\textasciigrave{}rows\textasciigrave{} must be a single number.")}
\NormalTok{  if (x==150) pass("✅ 150 rows.") else pass("ℹ️ Count recorded.")}
\NormalTok{\})}
\end{Highlighting}
\end{Shaded}

\subsection{Q10 --- Show column names}\label{q10-show-column-names}

Print the column names from iris.csv

\begin{Shaded}
\begin{Highlighting}[]
\NormalTok{\#| setup: true}
\NormalTok{\#| exercise: ex\_csv\_names}
\NormalTok{write.csv(iris, "iris.csv", row.names = FALSE)}
\end{Highlighting}
\end{Shaded}

\begin{Shaded}
\begin{Highlighting}[]
\NormalTok{\#| exercise: ex\_csv\_names}
\NormalTok{\#| exercise.lines: 4}
\NormalTok{\#| echo: false}
\NormalTok{\# Read the CSV and print the column names}
\NormalTok{df   \textless{}{-} read.csv("iris.csv")}
\NormalTok{cols \textless{}{-} }
\end{Highlighting}
\end{Shaded}

Use names.

\begin{solution}
\leavevmode

\begin{Shaded}
\begin{Highlighting}[]
\NormalTok{\#| exercise: ex\_csv\_names}
\NormalTok{\#| solution: true}
\NormalTok{\#| files: ["data/iris.csv"]   \# solution runs in its own sandbox — mount again}
\NormalTok{df   \textless{}{-} read.csv("iris.csv")}
\NormalTok{cols \textless{}{-} names(df)}
\NormalTok{cols}
\end{Highlighting}
\end{Shaded}

\end{solution}

\begin{Shaded}
\begin{Highlighting}[]
\NormalTok{\#| exercise: ex\_csv\_names}
\NormalTok{\#| check: true}
\NormalTok{gradethis::grade\_this(\{}
\NormalTok{  if (!exists("cols", envir=.envir\_result)) fail("Create \textasciigrave{}cols \textless{}{-} names(df)\textasciigrave{} and print it.")}
\NormalTok{  x \textless{}{-} get("cols", envir=.envir\_result)}
\NormalTok{  if (!is.character(x)) fail("\textasciigrave{}cols\textasciigrave{} should be a character vector from \textasciigrave{}names(df)\textasciigrave{}.")}
\NormalTok{  need \textless{}{-} c("Sepal.Length","Sepal.Width","Petal.Length","Petal.Width","Species")}
\NormalTok{  if (!all(need \%in\% x)) fail("Those don’t look like the iris columns—did you read the CSV and call \textasciigrave{}names(df)\textasciigrave{}?")}
\NormalTok{  pass("✅ Columns detected: Sepal/ Petal and Species.")}
\NormalTok{\})}
\end{Highlighting}
\end{Shaded}

\subsection{Q11 --- Print some of the
data}\label{q11-print-some-of-the-data}

Print the first 5 rows from iris.csv.

\begin{Shaded}
\begin{Highlighting}[]
\NormalTok{\#| setup: true}
\NormalTok{\#| exercise: ex\_csv\_head}
\NormalTok{write.csv(iris, "iris.csv", row.names = FALSE)}
\end{Highlighting}
\end{Shaded}

\begin{Shaded}
\begin{Highlighting}[]
\NormalTok{\#| exercise: ex\_csv\_head}
\NormalTok{\#| exercise.lines: 4}
\NormalTok{\#| echo: false}
\NormalTok{\# Read the CSV and print the column names}
\NormalTok{df   \textless{}{-} read.csv("iris.csv")}
\NormalTok{\#your code here}
\end{Highlighting}
\end{Shaded}

Use head.

\begin{solution}
\leavevmode

\begin{Shaded}
\begin{Highlighting}[]
\NormalTok{\#| exercise: ex\_csv\_head}
\NormalTok{\#| solution: true}
\NormalTok{\#| files: ["data/iris.csv"]   \# solution runs in its own sandbox — mount again}
\NormalTok{df \textless{}{-} read.csv("iris.csv")}
\NormalTok{head(df)}
\end{Highlighting}
\end{Shaded}

\end{solution}

\begin{Shaded}
\begin{Highlighting}[]
\NormalTok{\#| exercise: ex\_csv\_head}
\NormalTok{\#| check: true}
\NormalTok{gradethis::grade\_this(\{}
\NormalTok{  \# If the *last* expression is head(df, 5), .result should be a 5{-}row data.frame}
\NormalTok{  if (!(is.data.frame(.result) \&\& nrow(.result)==5)) \{}
\NormalTok{    fail("Show the first 5 rows (e.g., \textasciigrave{}head(df, 5)\textasciigrave{}) as the **last** line.")}
\NormalTok{  \}}
\NormalTok{  if (!all(c("Sepal.Length","Sepal.Width","Petal.Length","Petal.Width","Species") \%in\% names(.result))) \{}
\NormalTok{    fail("Your preview should be from \textasciigrave{}df\textasciigrave{} read from the CSV (it must have the iris columns).")}
\NormalTok{  \}}
\NormalTok{  pass("✅ Nice—now you can clearly see the columns and a few rows.")}
\NormalTok{\})}
\end{Highlighting}
\end{Shaded}

\subsection{Q12 --- Make a histogram from
iris.csv}\label{q12-make-a-histogram-from-iris.csv}

Make a histogram of the column Sepal.Length from iris.csv.

\begin{Shaded}
\begin{Highlighting}[]
\NormalTok{\#| setup: true}
\NormalTok{\#| exercise: ex\_csv\_hist}
\NormalTok{write.csv(iris, "iris.csv", row.names = FALSE)}
\end{Highlighting}
\end{Shaded}

\begin{Shaded}
\begin{Highlighting}[]
\NormalTok{\#| exercise: ex\_csv\_hist}
\NormalTok{\#| exercise.lines: 3}
\NormalTok{\#| echo: false}
\NormalTok{df \textless{}{-} read.csv("iris.csv")}
\end{Highlighting}
\end{Shaded}

Use hist (see Q5).

\begin{solution}
\leavevmode

\begin{Shaded}
\begin{Highlighting}[]
\NormalTok{\#| exercise: ex\_csv\_hist}
\NormalTok{\#| solution: true}
\NormalTok{\#| files: ["data/iris.csv"]   \# solution runs in its own sandbox — mount again}
\NormalTok{df \textless{}{-} read.csv("iris.csv")}
\NormalTok{hist(df$Sepal.Length, col = "lightgray", main = "Sepal.Length", xlab = "cm") }
\end{Highlighting}
\end{Shaded}

\end{solution}

\begin{Shaded}
\begin{Highlighting}[]
\NormalTok{\#| exercise: ex\_csv\_hist}
\NormalTok{\#| check: true}
\NormalTok{gradethis::grade\_this(\{}
\NormalTok{  if (!inherits(.result,"histogram")) fail("Use base \textasciigrave{}hist(df$Sepal.Length, ...)\textasciigrave{}.")}
\NormalTok{  if (!grepl("Sepal\textbackslash{}\textbackslash{}.Length", paste(.user\_code, collapse="\textbackslash{}n"))) fail("Plot \textasciigrave{}df$Sepal.Length\textasciigrave{}.")}
\NormalTok{  pass("✅ Histogram detected.")}
\NormalTok{\})}
\end{Highlighting}
\end{Shaded}

\subsection{Q13 --- Write a function using division}\label{q2}

Define a function called \texttt{divide\_nums} that takes two arguments
(\texttt{a} and \texttt{b}) and returns the results of:

\begin{enumerate}
\def\labelenumi{\arabic{enumi}.}
\tightlist
\item
  \texttt{a\ /\ b}
\item
  \texttt{b\ /\ a}
\end{enumerate}

Both results should be stored in separate variables before being
returned.

\begin{tcolorbox}[enhanced jigsaw, bottomtitle=1mm, arc=.35mm, toprule=.15mm, colframe=quarto-callout-note-color-frame, breakable, opacitybacktitle=0.6, leftrule=.75mm, titlerule=0mm, coltitle=black, opacityback=0, toptitle=1mm, bottomrule=.15mm, colbacktitle=quarto-callout-note-color!10!white, rightrule=.15mm, title=\textcolor{quarto-callout-note-color}{\faInfo}\hspace{0.5em}{Info}, left=2mm, colback=white]

Remember: in R, division by zero (1/0) will return Inf rather than an
error.

\end{tcolorbox}

\begin{tcolorbox}[enhanced jigsaw, bottomtitle=1mm, arc=.35mm, toprule=.15mm, colframe=quarto-callout-note-color-frame, breakable, opacitybacktitle=0.6, leftrule=.75mm, titlerule=0mm, coltitle=black, opacityback=0, toptitle=1mm, bottomrule=.15mm, colbacktitle=quarto-callout-note-color!10!white, rightrule=.15mm, title=\textcolor{quarto-callout-note-color}{\faInfo}\hspace{0.5em}{Info}, left=2mm, colback=white]

In R, lists are written with list(a, b, c). Each element can be
anything: a number, a string, or even another list.

\end{tcolorbox}

\begin{Shaded}
\begin{Highlighting}[]
\NormalTok{\#| exercise: ex\_fun\_div}
\NormalTok{\#| exercise.lines: 8}
\NormalTok{\#| echo: false}
\NormalTok{divide\_nums \textless{}{-} function(a, b) \{}
\NormalTok{  \# your code here}
\NormalTok{\}}
\end{Highlighting}
\end{Shaded}

Start with: res1 \textless- a / b res2 \textless- b / a and then use the
list command.

\begin{solution}
\leavevmode

\begin{Shaded}
\begin{Highlighting}[]
\NormalTok{\#| exercise: ex\_fun\_div}
\NormalTok{\#| solution: true}
\NormalTok{divide\_nums \textless{}{-} function(a, b) \{}
\NormalTok{  res1 \textless{}{-} a / b}
\NormalTok{  res2 \textless{}{-} b / a}
\NormalTok{  list(res1, res2)}
\NormalTok{\}}
\end{Highlighting}
\end{Shaded}

\end{solution}

\begin{Shaded}
\begin{Highlighting}[]
\NormalTok{\#| exercise: ex\_fun\_div}
\NormalTok{\#| check: true}
\NormalTok{gradethis::grade\_this(\{}
\NormalTok{  code \textless{}{-} .user\_code}

\NormalTok{  \# 0) Empty/untouched?}
\NormalTok{  if (!nzchar(gsub("\textbackslash{}\textbackslash{}s|\#.*", "", code))) \{}
\NormalTok{    fail("Type your answer in the editor.")}
\NormalTok{  \}}

\NormalTok{  \# 1) Find the student\textquotesingle{}s env (webR/gradethis names vary)}
\NormalTok{  env \textless{}{-} get0(".envir\_result", ifnotfound = NULL)}
\NormalTok{  if (is.null(env)) env \textless{}{-} get0(".user\_env", ifnotfound = NULL)}
\NormalTok{  if (is.null(env)) env \textless{}{-} parent.frame()}

\NormalTok{  \# 2) Get their function (prefer the named binding; fall back to .result)}
\NormalTok{  f \textless{}{-} NULL}
\NormalTok{  if (exists("divide\_nums", envir = env, inherits = FALSE)) \{}
\NormalTok{    f \textless{}{-} get("divide\_nums", envir = env)}
\NormalTok{  \} else if (is.function(.result)) \{}
\NormalTok{    f \textless{}{-} .result}
\NormalTok{  \}}

\NormalTok{  if (!is.function(f)) \{}
\NormalTok{    fail("Define a function named \textasciigrave{}divide\_nums(a, b)\textasciigrave{} (use \textasciigrave{}divide\_nums \textless{}{-} function(a, b) \{ ... \}\textasciigrave{}).")}
\NormalTok{  \}}

\NormalTok{  \# 3) Run it and grade}
\NormalTok{  val \textless{}{-} tryCatch(f(8, 2), error = function(e) e)}
\NormalTok{  if (inherits(val, "error")) \{}
\NormalTok{    msg \textless{}{-} conditionMessage(val)}
\NormalTok{    if (grepl("object .* not found", msg)) \{}
\NormalTok{      fail("Looks like you referenced a variable that wasn’t created (e.g., \textasciigrave{}res1\textasciigrave{} or \textasciigrave{}res2\textasciigrave{}).")}
\NormalTok{    \} else \{}
\NormalTok{      fail(paste0("Your function raised an error: ", msg))}
\NormalTok{    \}}
\NormalTok{  \}}

\NormalTok{  if (!is.list(val) || length(val) != 2) \{}
\NormalTok{    fail("Return a **list** of two elements: \textasciigrave{}list(res1, res2)\textasciigrave{}.")}
\NormalTok{  \}}
\NormalTok{  if (!(is.numeric(val[[1]]) \&\& is.numeric(val[[2]]))) \{}
\NormalTok{    fail("Both returned values must be numeric (\textasciigrave{}a/b\textasciigrave{} and \textasciigrave{}b/a\textasciigrave{}).")}
\NormalTok{  \}}

\NormalTok{  ok \textless{}{-} isTRUE(all.equal(val[[1]], 4,    tol = 1e{-}8)) \&\&}
\NormalTok{        isTRUE(all.equal(val[[2]], 0.25, tol = 1e{-}8))}
\NormalTok{  if (!ok) \{}
\NormalTok{    fail("Close! Compute \textasciigrave{}res1 \textless{}{-} a / b\textasciigrave{} and \textasciigrave{}res2 \textless{}{-} b / a\textasciigrave{}, then return \textasciigrave{}list(res1, res2)\textasciigrave{}.")}
\NormalTok{  \}}

\NormalTok{  pass("✅ Well done! Function name, outputs, and return format are correct.")}
\NormalTok{\})}
\end{Highlighting}
\end{Shaded}





\end{document}
